\documentclass[12pt,a4paper]{article}
\usepackage[utf8x]{inputenc}
\usepackage{ucs}
\usepackage[MeX]{polski}
\usepackage{fancyhdr}
\usepackage{amsmath}
\usepackage{amsfonts}
\usepackage{amssymb}
\pagestyle{fancy}
\usepackage{enumerate}
\usepackage{listings}
\usepackage{subfig}

\usepackage{graphicx}
\usepackage{hyperref}
\usepackage{datetime}

\renewcommand{\maketitle}{\begin{titlepage}
\begin{center}

\vspace*{3cm}
\noindent \rule{\linewidth}{0.4mm}
\LARGE \textsc{Założenia projektu Modułu Antykradzieżowego}\\
\vspace*{0.5cm}
\rule{\linewidth}{0.4mm}
\large \textsc{Projekt realizowany w ramach kursu Sterowniki Robotów na Politechnice Wrocławskiej}
\vspace*{2cm}
\large\flushleft
\textsc{\textbf{Nazwa kursu:} Sterowniki Robotów}\\
\textsc{\textbf{Nazwa projektu:} Moduł Antykradzieżowy SMS }\\ 
\textsc{\textbf{Akronim projektu:} MASS }\\ 
\textsc{\textbf{Termin zajęć:} środa TP 11:15}\\
\textsc{\textbf{Autor 0:} Maciej Flis 218***} \\ 
\textsc{\textbf{Autor 1:} Krzysztof Dąbek 218549} \\ 
\textsc{\textbf{Numer grupy:} 5}\\
\textsc{\textbf{Prowadzący:} mgr inż Wojciech Domski}

\vspace*{4cm}
\centering
\textsc{\today}\\
\end{center}
\end{titlepage}
\newpage
}

\begin{document}
\maketitle
\normalsize

\section{Główne założenia projektowe}
\begin{itemize}
\item Stworzenie software'owego projektu modułu antykradzieżowego na gotowej płytce, zaprojektowanej przez Macieja Flisa, opartej na mikrokontrolerze STM32F746VGT6.
\item Napisanie programu z użyciem biblioteki HAL generowanej przez dedykowane IDE od firmy ST (STM32 toolchain: STM32CubeMX, Atollic Truestudio).
\item Wgrywanie programu na płytkę za pomocą konwertera USB $\rightarrow$ UART.
\item Wykorzystanie modułu GPS do uzyskania lokalizacji płytki.
\item Wykorzystanie akcelerometru do uruchamiania wszystkich modułów w momencie poruszenia płytką.
\item Przesyłanie informacji na żądanie w wiadomości SMS z użyciem modułu GSM.
\item Emulacja pamięci EPROM w pamięci flash za pomocą bufora cyklicznego i zapisywanie w niej kolejnych lokalizacji modułu (po każdym przejściu w stan czuwania oraz co ustalony czas w stanie aktywnym).
\item Uruchomienie zegara czasu rzeczywistego (RTC) i zapisywanie informacji o czasie przeprowadzenia pomiaru wraz z lokalizacją w symulowanej pamięci EPROM w formie logu.
\item Wysyłanie danych z logu do komputera przez port USB.
\end{itemize}

\section{Harmonogram}
\begin{itemize}
\item (20.03. - 26.03.) Konfiguracja peryferiów mikrokontrolera, wygenerowanie kodu, wgranie programu na płytkę przez UART.
\item (27.03. - 2.04.) Oprogramowanie komunikacji USB i akcelerometru.
\item (3.04. - 23.04.) Uruchomienie modułu GSM, uruchomienie modułu GPS.
\item (24.04. - 7.05.) Oprogramowanie RTC, emulacja pamięci EPROM.
\item (8.05. - 14.05.) Juwenalja :)
\item (15.05. - 26.05) Przygotowanie raportu, przedstawienie pierwszego etapu prac nad projektem.
\item (27.05. - 11.06) Przygotowanie gotowej aplikacji, testy, rozwiązywanie problemów.
\item (12.06. - 22.06) Przygotowanie dokumentacji technicznej, przedstawienie gotowego projektu.
\end{itemize}

\tableofcontents

\section{Bibliografia}
\url{https://edu.domski.pl/kursy/sterowniki-robotow/sr-projekt/}

\end{document}